\documentclass[a4paper,papersize,12pt]{jsarticle}
\usepackage{./conf/jefregulation}

% textlint-disable
\documentnumber{{\color{red}創}第{\color{red}0}号}
\author{\color{red}運営役員会長}
\title{\color{red}文 書 取 扱 規 程}
\date{
    {
        \color{red}
        令和 1年 1月 1日 制  定 (創第0号)\\
        令和 2年 2月 1日 一部改定 (創第0号)
    }
}
% textlint-enable

\begin{document}
\maketitle

\begin{mokuji}
\shou{総則}
\shou{文書番号}
\fusoku
\end{mokuji}

% textlint-disable-next-line ja-no-successive-word, no-doubled-conjunction, max-kanji-continuous-len
\shou{総   則}
\jou[(目的)]{この規程は、別に定めるもののほか、共有文書の作成について必要な事項を定めるものとする。}

\jou[(用語の定義)]{この規程において、次の各号に掲げる用語の意義は、それぞれ当該各号に定めるところによる。}
\gou{共有文書 会の会員が作成した文書、図画、写真、フィルム及び電磁的記録(電子的方式、磁気的方式その他人の知覚によっては
認識することができない方式で作られた記録をいう。以下同じ。)であって、当該会の会員が組織的に用いるものとして、当該会が保有しているものをいう。}
\gou{会 会規程(令和 3 年創第 2 号)第 1 条に規定される会をいう。}

\newpage
\shou{文 書 番 号}
\jou{共有文書には、記号及び文書番号を付さなければならない。ただし、共有文書で軽易なものにあっては、この限りでない。}

\jou{共有文書には、次に掲げる会及び役職等の記号、番号の順で文書番号を記載すること。}
\gou{記号は、次の例示によること。ただし、これにより難いときは、運営役員会の会長に合議の上、別に定めることができる。}
% textlint-disable
\begin{enumerate}[label=\aiu*]
    \item 内部共有文書\\
        創第    号(運営役員会長)\\
        幹第    号(幹部会)\\
        幹運第   号(運営役員会)
    \item 外部共有文書 内部共有文書の記号に「日」を冠用すること。
\end{enumerate}
% textlint-enable
\gou{番号は、次に掲げるところにより付すること。}
% textlint-disable
\begin{enumerate}[label=\aiu*]
    \item 年度毎に共有する順序に従い一連の番号を付すること。
    \item 文書の作成毎に付番し、1つの案件については、原則として、同一番号とすること。
\end{enumerate}
% textlint-enable

\newpage
% textlint-disable
\begin{fusoku}
\end{fusoku}
\begin{fusoku}[令和3年2月5日創第0号]
\end{fusoku}
\begin{fusokushou}
\end{fusokushou}
\begin{fusokushou}[令和3年2月5日創第0号]
\heading[施行期日]
\kou{この規程は、公布の日から施行する。}
\heading[経過措置]
\kou{改正後の文書取扱規程の規定は、この規定の施行の日以後に作成する共有文書について適用し、
施行日前に作成した文書、フィルム及び電磁的記録については、なお従前の例による。}
\sakujo{\kou}
\end{fusokushou}
% textlint-enable

\besshi[1]
\besshi[2]
\besshirimen[2]
\youshiki
\youshiki[1-1]
\youshiki[2]
\youshikirimen[2]
\end{document}
